\documentclass[11pt, oneside]{article}
\usepackage[top=1.5in, bottom=1.5in]{geometry}
\usepackage{fancyhdr}
\usepackage{setspace}
\usepackage{mdwlist}
\usepackage{comment}
% Reworking Headers
\pagestyle{fancy}
\fancyhf{}
\lhead{CS 469 : Security Engineering }
\rhead{\thepage}

\title{``iSESnapchat'' Proposal}
\author{Cameron Pelkey \and Mike Brooks \and John Reynolds}
\date{}

\begin{document}
\maketitle
%\doublespacing

\section{Snapchat}
The mobile application ``Snapchat'' has become very popular not only for its ability to quickly and securely send image and video, but also for the novel approach in which it does so. Within the application, users cannot export images or video received and, once viewed, the data persists on the device for only a set length of time after which it is permanently deleted. However, there is one critical flaw in this scheme: screenshots of received messages may be taken by software imbedded within the mobile device itself. In this manner, users are able to easily bypass the guaranteed security of the application.

As an application, Snapchat may not intercept the hardware interrupts that that allow the taking of screenshots on mobile devices. All that the application is capable of doing is notifying the sending-party that a screenshot of their once-fleeting message was taken.

On the Mac platform, there are already numerous ways for users to send and receive images, whether it be one of several multimedia-handling chat services or simply email. In all of these cases, a user has absolutely no expectation of control over the message once it has been sent.

No application presently available on the Mac App Store provides the functionality of a Snapchat-like application.

\section{Application Prototype Description}
Students Pelkey, Brooks, and Reynolds propose to develop a prototype for a more secure Snapchat-like service capable of sending, receiving, and viewing time-sensitive images.

The application will:
\begin{itemize*}
	\item {Present the user with a means of capturing or selecting an image for transfer.}
	\item {Allow the user to select one or more recipients of this image from a list of additional users.}
	\item {Encrypt this image for secure transfer using an as-yet-to-be-determined crypto scheme.}
	\item {Display the image to the recipient using a custom-built viewer so as to prevent the recipient from taking screenshots or otherwise saving this image to the disk.}
	\item {Erase all image data once the allowed viewing time has elapsed.}
\end{itemize*} 

\begin{comment}
The application would present to the user a method of capturing or selecting an image to send to another user. The user would then be prompted for the ``destination-user'', who upon receiving the image would be able to see the image for some set amount of time. After the message is viewed, the message deletes itself. The custom-built viewer window displayed on the screen to view the message would be made to prevent screenshots. All transmissions would be encrypted using an as-yet-to-be-determined crypto scheme. 
\end{comment}

\newpage
Our goal is not to create new methods of encryption or image capture; libraries already exist which provide this functionality. Furthermore, in the interest of security it is best practice to implement a known, tested crypto suite. The novelty in the proposed approach is to prevent data leakage from the application. Ideally, there will be no way to extract an image from the application, to read data in motion, or to prevent the deletion of a message. 

\section{Capabilities and Facilities of Project Members}
Prior to this program, each project member has had experience in applications development. The combined skill set of the group is well-suited for a project of this nature; some members have experience in the implementation of cryptographic methods in communication while others have practical knowledge of open source imaging libraries, such as OpenCV.

\section{Stretch Goals}
Time permitting, additional functionality would be considered. As an implementation detail, our ``secure'' window which prevents screenshots would need a rendering backend to run. This is a trivial task for picture rendering, however video may be a challenge. Research would be done for methods of capturing, transferring, and rendering video in our secure window.

Additionally, other filetypes could be rendered, allowing for file sharing of arbitrary file types. The use case of being able to send one-time-viewable documents could also be valuable for an end user.

\end{document}